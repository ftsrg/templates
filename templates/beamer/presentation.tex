\documentclass[t,aspectratio=169]{beamer}
\usepackage{ftsrgpresentation}

\usepackage[english]{babel}
\usepackage{xstring} % you probably dont need this
\newcommand{\extractFontName}[1]{%
  \StrBetween{#1}{[}{]}
}

\title{Title of the Presentation}
\subtitle{Venue'YY}
\author{Előadó Elemér, Társszerző Tamás}
\supervisor{Supervisor: Konzulens Károly}

\begin{document}
	{
		\setbeamertemplate{footline}{} 
		\begin{frame}
			\titlepage
		\end{frame}
	}
	\addtocounter{framenumber}{-1}
  
\begin{frame}{Example frame}
  \begin{itemize}
  \item Write your points here
  \item You can use subitems
    \begin{itemize}
    \item And sub-subitems
      \begin{itemize}
      \item But please don't go any lower
      \end{itemize}
    \end{itemize}
  \item You must have \emph{opensans} installed. The current font is: 
  \begin{itemize}
      \item \extractFontName{\expandafter\meaning\the\font}
  \end{itemize}

  \item The attached \texttt{.latexmkrc} helps you
    \begin{itemize}
    \item Just run \texttt{latexmk presentation}
    \end{itemize}
  \item<2-> Animations are also supported\tikzmark{bubblehere}
  \end{itemize}
  \callout<3>[text width=5cm]{bubblehere}{Use \texttt{\textbackslash
      tikzmark} and \texttt{\textbackslash callout} to place ``speech
    bubbles''}
    
  \accentcallout<3>[text width=5cm]{bubblehere}(150:0.5cm){Use \texttt{\textbackslash accentcallout} to place accent-colored ``speech bubbles''}[AccentPurple]
\end{frame}

\begin{frame}[c]{Figures}
  \begin{figure}
    \centering
    \includegraphics[width=\textwidth]{figures/BME.pdf}
    \caption{Write your caption here}
  \end{figure}
\end{frame}

\begin{frame}[c]{Vector graphics}
  \centering
  \begin{tikzpicture}[
    remember picture,
    workflow item/.style={ftsrg bubble, minimum height=1.5cm,
      text width=3cm},
    workflow subitem/.style={workflow item, minimum height=1.05cm,
      text width=2.1cm, font=\footnotesize, fill=orange},
    workflow arrow tip/.tip={Stealth[scale=1.2]},
    workflow arrow/.style={ultra thick,-{workflow arrow tip},font=\footnotesize},
    workflow bidirectional arrow/.style={workflow arrow,
      {workflow arrow tip}-{workflow arrow tip}},
    ]
    \node[workflow item, minimum height=2.2cm, text width=4.4cm] (tikz) {%
      With Ti\emph{k}Z commands };
    \node[workflow item, left=of tikz] (figures) { Figures };
    \node[workflow item, above=of figures, alt=<2->{fill=orange}{}]
    (draw) { Draw };
    \node[workflow item, below=of tikz] (anim) { \only<2->{Animations} };
    \draw[workflow arrow] (draw) edge (figures);
    \draw[workflow arrow] (figures) edge (tikz);
    \draw[workflow bidirectional  arrow,transform canvas={xshift=-0.5cm}]
    (tikz) edge (anim);
    \draw[workflow bidirectional  arrow,transform canvas={xshift=0.5cm}]
    (tikz) edge (anim);
    \draw[workflow arrow,dashed,visible on=<2->]
    (figures) |- node[at end,above,anchor=south east,xshift=-0.6cm]
    {You can also add} (anim);
  \end{tikzpicture}
  \callout<2->[xshift=-1cm]{draw.east}(150:1.5cm){%
    You can place callouts over figures, too}
\end{frame}

\begin{demoframe}{Show your software}
  Use the \texttt{demoframe} environment to create this slide
\end{demoframe}

\begin{frame}{Equations}
  \begin{align*}
    i \hbar \frac{\partial}{\partial t} \Psi (\mathbf{r}, t)
    &= \hat{H} \Psi (\mathbf{r}, t) \\
    &= \mleft[ \frac{- \hbar^2}{2 \mu} \nabla^2 + V(\mathbf{r}, t)
      \mright] \Psi (\mathbf{r}, t)
  \end{align*}

  From
  \href{https://en.wikipedia.org/wiki/Schr\%C3\%B6dinger_equation}{%
  Wikipedia, the free encyclopedia}
\end{frame}

\begin{frame}{Columns and blocks}
  \begin{columns}
    \begin{column}{0.49\linewidth}
      \begin{block}{Block with a title}
        This is a block.\strut
      \end{block}

      \begin{theorem}
        $\forall b \in \textit{Blocks} : b$ is a cool block
      \end{theorem}
    \end{column}

    \begin{column}{0.49\linewidth}
      \begin{alertblock}{Alerted block}
        Achtung! \alert{Achtung!}\strut
      \end{alertblock}

      \begin{exampleblock}{Example block}
        This is an example of an example block. A meta-example block.
      \end{exampleblock}
    \end{column}
  \end{columns}

  \vspace{1ex}

  \begin{block}{}
    A block which spans two columns and has no title.
  \end{block}
  \vspace{1ex}
  \begin{coloredblock}{Custom colored blocks can be also used}{bg=ftsrg@AccentOrange,fg=white}{bg=ftsrg@AccentOrange!25,fg=ftsrg@DarkGray}
    Use \texttt{ftsrg@Accent<Color>} for titles, and \texttt{ftsrg@Accent<Color>!25} for bodies.
  \end{coloredblock}
\end{frame}

\newcounter{rows}
\newcounter{cols}[rows]

\begin{frame}[c]{Fonts (series vs. shape)}
    \centering
    
    \setcounter{rows}{0}
    \setcounter{cols}{0}

    \def\shapes{n, it, sl}
    \def\series{l, lc, m, sb, b, bx, bc, eb}
    
    \begin{tikzpicture}[scale=0.8,x=5cm,y=0.8cm,every node/.style={transform shape}]
                \tikzset{vertex/.style = {shape=rectangle,text width=,minimum height=0.7cm,minimum width=4.75cm}}
                
                \foreach \fshape in \shapes
                {
                   \node[vertex,fill=white,draw=white] (c) at  (\thecols, 1) {\scriptsize \fshape};
                   \stepcounter{cols}
                }
                \setcounter{cols}{0}
                
                \foreach \fseries in \series
                {
                   \node[vertex,fill=white,draw=white,minimum width=0] (c) at  (-0.6 , -\therows) {\scriptsize \fseries};
                
                   \foreach \fshape in \shapes {

                        \node[vertex,fill=white,draw=black] (c) at  (\thecols, -\therows) {\footnotesize\fontseries{\fseries}\fontshape{\fshape}\selectfont{\extractFontName{\expandafter\meaning\the\font}}};
                        \stepcounter{cols}
                    
                    }
                    \stepcounter{rows}
                }
        
    \end{tikzpicture}
    \begin{alertblock}{}
    \centering
        If not all fonts match the regex \texttt{OpenSans-.*\textbackslash.ttf}, install the missing fonts!
    \end{alertblock}
    
\end{frame}

\begin{frame}[c]{Colors}
    \centering
    
    \setcounter{rows}{0}
    \setcounter{cols}{0}

    \def\colors{ftsrg@Gray, ftsrg@DarkGray, ftsrg@AccentBlue, ftsrg@AccentRed, ftsrg@AccentPurple, ftsrg@AccentOrange, ftsrg@AccentLightBlue, ftsrg@AccentGreen}
    \def\shades{!20!black, !40!black, !60!black, !80!black, !100, !80, !60, !40, !20}
    
    \begin{tikzpicture}[scale=0.8,x=1.5cm,y=0.8cm,every node/.style={transform shape}]
                \tikzset{vertex/.style = {shape=rectangle,text width=,minimum height=0.5cm,minimum width=0.5cm}}
                
                \foreach \cshade in \shades
                {
                   \node[vertex,fill=white,draw=white] (c) at  (\thecols, 1) {\scriptsize \cshade};
                   \stepcounter{cols}
                }
                \setcounter{cols}{0}
                
                \foreach \ftsrgcolor in \colors
                {
                   \node[vertex,fill=white,draw=white] (c) at  (-1.35 , -\therows) {\scriptsize \ftsrgcolor};
                
                   \foreach \shade in \shades {

                        \node[vertex,fill=\ftsrgcolor\shade,draw=black] (c) at  (\thecols, -\therows) {};
                        \stepcounter{cols}
                    
                    }
                    \stepcounter{rows}
                }
        
    \end{tikzpicture}
    
\end{frame}

\begin{frame}[c]{}
  \centering {\LARGE Have fun with Beamer!}
\end{frame}

\end{document}
