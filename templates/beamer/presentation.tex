\documentclass[t,aspectratio=169]{beamer}
\usepackage{ftsrgpresentation}

\usepackage[english]{babel}

\title{Title of the Presentation}
\subtitle{Venue'YY}
\author{Előadó Elemér, Társszerző Tamás}
\supervisor{Supervisor: Konzulens Károly}
\university{Budapest University of Technology and Economics}
\department{Department of Measurement and Information Systems}
\researchgroup{ftsrg Research Group}

\begin{document}
	{
		\setbeamertemplate{footline}{} 
		\begin{frame}
			\titlepage
		\end{frame}
	}
	\addtocounter{framenumber}{-1}
  
\begin{frame}{Example frame}
  \begin{itemize}
  \item Write your points here
  \item You can use subitems
    \begin{itemize}
    \item And sub-subitems
      \begin{itemize}
      \item But please don't go any lower
      \end{itemize}
    \end{itemize}
  \item You must compile this presenation with Lua\LaTeX\\
    and have \emph{Calibri}, \emph{Cambria} and \emph{Cambria Math}
    installed on you computer as OpenType fonts
  \item The attached \texttt{.latexmkrc} helps you
    \begin{itemize}
    \item Just run \texttt{latexmk presentation}
    \end{itemize}
  \item<2-> Animations are also supported\tikzmark{bubblehere}
  \end{itemize}
  \callout<3>[text width=5cm]{bubblehere}{Use \texttt{\textbackslash
      tikzmark} and \texttt{\textbackslash callout} to place ``speech
    bubbles''}
\end{frame}

\begin{frame}[c]{Figures}
  \begin{figure}
    \centering
    \includegraphics[width=\textwidth]{figures/BME.pdf}
    \caption{Write your caption here}
  \end{figure}
\end{frame}

\begin{frame}[c]{Vector graphics}
  \centering
  \begin{tikzpicture}[
    remember picture,
    workflow item/.style={ftsrg bubble, minimum height=1.5cm,
      text width=3cm},
    workflow subitem/.style={workflow item, minimum height=1.05cm,
      text width=2.1cm, font=\footnotesize, fill=orange},
    workflow arrow tip/.tip={Stealth[scale=1.2]},
    workflow arrow/.style={ultra thick,-{workflow arrow tip},font=\footnotesize},
    workflow bidirectional arrow/.style={workflow arrow,
      {workflow arrow tip}-{workflow arrow tip}},
    ]
    \node[workflow item, minimum height=2.2cm, text width=4.4cm] (tikz) {%
      With Ti\emph{k}Z commands };
    \node[workflow item, left=of tikz] (figures) { Figures };
    \node[workflow item, above=of figures, alt=<2->{fill=orange}{}]
    (draw) { Draw };
    \node[workflow item, below=of tikz] (anim) { \only<2->{Animations} };
    \draw[workflow arrow] (draw) edge (figures);
    \draw[workflow arrow] (figures) edge (tikz);
    \draw[workflow bidirectional  arrow,transform canvas={xshift=-0.5cm}]
    (tikz) edge (anim);
    \draw[workflow bidirectional  arrow,transform canvas={xshift=0.5cm}]
    (tikz) edge (anim);
    \draw[workflow arrow,dashed,visible on=<2->]
    (figures) |- node[at end,above,anchor=south east,xshift=-0.6cm]
    {You can also add} (anim);
  \end{tikzpicture}
  \callout<2->[xshift=-1cm]{draw.east}(180:1.5cm){%
    You can place callouts over figures, too}
\end{frame}

\begin{demoframe}{Show your software}
  Use the \texttt{demoframe} environment to create this slide
\end{demoframe}

\begin{frame}{Equations}
  \begin{align*}
    i \hbar \frac{\partial}{\partial t} \Psi (\mathbf{r}, t)
    &= \hat{H} \Psi (\mathbf{r}, t) \\
    &= \mleft[ \frac{- \hbar^2}{2 \mu} \nabla^2 + V(\mathbf{r}, t)
      \mright] \Psi (\mathbf{r}, t)
  \end{align*}

  From
  \href{https://en.wikipedia.org/wiki/Schr\%C3\%B6dinger_equation}{%
  Wikipedia, the free encyclopedia}
\end{frame}

\begin{frame}{Columns and blocks}
  \begin{columns}
    \begin{column}{0.49\linewidth}
      \begin{block}{Block with a title}
        This is a block.\strut
      \end{block}

      \begin{theorem}
        $\forall b \in \textit{Blocks} : b$ is a cool block
      \end{theorem}
    \end{column}

    \begin{column}{0.49\linewidth}
      \begin{alertblock}{Alerted block}
        Achtung! \alert{Achtung!}\strut
      \end{alertblock}

      \begin{exampleblock}{Example block}
        This is an example of an example block.\\
        A meta-exaple block.
      \end{exampleblock}
    \end{column}
  \end{columns}

  \vspace{1ex}

  \begin{block}{}
    A block which spans two columns and has no title.
  \end{block}
\end{frame}

\begin{frame}[c]{}
  \centering {\LARGE Have fun with Beamer!}
\end{frame}

\end{document}
