% Hazi Feladat / Meresi jegyzokony sablon BME MIT
% Keszult: 2012.13.17
% Leiras: Ebbe a fajlba kerul a lenyegi resz, a szoveg. A legfelsobb szintu felsorolas a section (chapter nem hasznalatos).

\section{Bevezetés}
\subsection{A feladat}
Az alábbi programokkal dolgoztam:
\begin{enumerate}
\item Debian Linux 2.6.32-5-amd64
\item Ez egy pelda felsorolas
\end{enumerate}

\section{Felhasználói dokumentáció}
\subsection{Feladat 1}
\subsubsection{Alfeladat 1}
Az alábbi script segítségével minden fájlnak köszönök az aktuális mappában.
\begin{lstlisting}[frame=single,float=!ht]
ls | while read i; do echo Szia $i; done
\end{lstlisting}
\subsubsection{Alfeladat 2}
\begin{figure}[h]
\begin{center}
\includegraphics[height=2cm]{figures/BMElogo.png}
\caption{A BME logója}
\label{fig:BMElogo}
\end{center}
\end{figure}

\section{Fejlesztői dokumentáció}
\subsection{Feladat 1}
Álljon itt egy lista vala.
\begin{itemize}
 \item \LaTeX
 \item Foo, olvasható Lamport cikkéből. \cite{lamport94}
\end{itemize}
\begin{equation}
x_1 = \frac{5 + \sqrt{25 - 4 \times 6}}{2} = 3
\end{equation}
 

\subsection{Feladat 2}
Ahogyan azt a \ref{fig:BMElogo}. ábrán láttuk...

